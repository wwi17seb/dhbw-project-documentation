\section{Lessons Learned}
Während der Arbeit an dem Projekt wurden viele Erkenntnisse gewonnen, neues Wissen erlangt sowie wertvolle Erfahrungen gesammelt.
Zunächst lässt sich festhalten, dass die Projektmitglieder viel technische als auch organisatorische Aspekte gelernt haben.
Im Folgenden werden drei Lessons Learned näher erläutert, die Optimierungspotenziale während oder nach dem Projekt bereitgehalten haben.
\subsection{Kommunikation}
Eine Herausforderung, die in jedem Gruppenprojekt zu meistern ist, ist die Kommunikation zwischen den Projektmitgliedern erfolgreich zu gestalten. 
Hierbei war eine besondere Hürde auch die Gruppengröße, die ein zusätzlicher Faktor für die Koordination und die Absprachen war. 
Ein respektvoller Umgang zwischen den Projektmitgliedern ist sehr wichtig und beeinflusst den Projektverlauf.
Während des Projektverlaufs wurde die Erkenntnis gezogen, dass eine verstärkte Kommunikation in den Teams sich positiv auf den Fortschritt auswirkt. 
Diese Optimierungsmöglichkeit wurde nach einer Reflexion des 5. Semesters gezogen und somit im 6. Semester verbessert. 

\subsection{Inhaltliche Ziele mit festen Milestones}
Wie bereits in der \hyperref[ch:zeitplanung]{Zeitplanung} ersichtlich, wurden in dem 6. Semester auch inhaltliche Ziele festgelegt. 
Diese Vorgehensweise hat sich als zielführender herausgestellt, da der Projektverlauf dadurch besser kontrolliert sowie die Umsetzung besser geplant werden konnte.
Auch für das 5. Semester wäre dies von Vorteil gewesen und wird als Erfahrung aus dem Projekt mitgenommen.

\subsection{Punktesystem}
Zur Bewertung der Leistung der Projektmitglieder wurde der Aufwand als Ist-Stunden verwendet sowie vergütet.
Da sich das Projektteam aus Personen mit unterschiedlichen Kompetenzen zusammensetzt, wurde oft in Kleingruppen zusammengearbeitet. 
Hierbei haben unter anderem Mitglieder mit höherer Expertise anderen Projektmitgliederns ausgeholfen.
Beide Personen haben jedoch den gleichen Aufwand bekommen, sodass dies von den Mitgliedern mit tiefergreifendem Wissensstand als unfaire Lösung angesehen wurde. 
Zusätzlich hat sich im Laufe des Projekts herausgestellt, dass es Projektmitglieder gibt, die das Projektergebnis maßgeblich vorangetrieben haben.
Um diese Leistung sowie die eingebrachte Expertise zu honorieren, wäre es möglich ein Bonussystem einzuführen. 
Dieses würde es ermöglichen, den entsprechenden Personen einen Bonus zu vergeben und entsprechend die Leistung zu vergüten.

Das Bewertungssystem hat sich während des  Projekts als strittiger Punkt herausgestellt, da es trotz Änderungen von einigen als unfair empfunden wird.
Dort ist allerdings festzuhalten, dass es besonders bei einer solchen Gruppengröße, sehr herausfordernd ist es allen Recht zu machen.