\section{Erfüllung der Anforderungen}
Durch die nachfolgende Tabelle werden die Anforderungen an das Projekt zusammengefasst
und der Umsetzung gegenübergestellt sowie anhand ihrer Erfüllung durch die implementierte Webanwendung bewertet.

\begin{longtable}[H]{|p{2,8cm}|p{4,2cm}|p{6,7cm}|}		
	\multicolumn{3}{|r|}{\textit{Fortsetzung nächste Seite}} \\ \hline
	\endfoot
	\endlastfoot
	\hline &&\\[-0.5em]
	\textbf{Anforderung} & \head{Kurztitel} & \head{Umsetzung} \\ \hline
	\endfirsthead
	\hline &&\\[-0.5em]
	\textbf{Anforderung} & \head{Kurztitel} & \head{Umsetzung} \\ \hline
	\endhead
	\parbox[t]{3cm}{\textbf{A1}} & Dozentenpool & Die Webanwendung verfügt über den Navigationspunkt \textit{Dozenten}, unter welchem die Dozentenverwaltung mit der zugrundeliegenden \textit{PostgreSQL} Datenbank stattfindet. Diese als \textit{Muss} priorisierte Anforderung wurde damit erfüllt.\\ \hline %Muss
	\parbox[t]{3cm}{\textbf{A2}} & Stundenplan &  Das Erzeugen und Verwalten von Kursen und Vorlesungen ist auf der Hauptansicht der Anwendung möglich. Diese als \textit{Muss} priorisierte Anforderung wurde ebenfalls erfüllt.\\ \hline  %Muss
	\parbox[t]{3cm}{\textbf{A3}} & Google Calendar & Unter Verwendung des React Schedulers konnte Google Calendar angebunden und integriert werden, sodass auch diese \textit{Muss}-Anforderung erfüllt wurde. \\ \hline %Muss
	\parbox[t]{3cm}	{\textbf{A4}} & Profilzuordnung & Die Benutzer der Software können sich wie angefordert mit einer Mail-Adresse registrieren. Es besteht darüberhinaus die Möglichkeit, sich mit einem beliebigen Benutzernamen zu registrieren. Diese \textit{Muss}-Anforderung wurde somit umgesetzt.\\ \hline %Muss
	\parbox[t]{3cm}	{\textbf{A5}} & Modulkataloge & Unter dem Navigationspunkt \textit{Modulkatalog} der Anwendung können Modulkataloge verschiedener Studienrichtungen verwaltet werden. Auch diese \textit{Muss}-Anforderung wurde erfüllt.\\ \hline %Muss
	\parbox[t]{3cm}	{\textbf{A6}} & Kurskoordination & Durch die Umsetzung der Beziehung \enquote{Kurs-zu-Studiengangsleiter} in der zugrundeliegenden relationalen Datenbank wird diese Anforderung, dass Studiengangsleiter mehrere Kurse koordinieren können, voll umgesetzt.\\ \hline %Muss
	\parbox[t]{3cm}	{\textbf{A7}} & Eindeutigkeit der Module & Die eindeutige Identifizierbarkeit von Modulen wird mithilfe der Entität für Modulgruppen sowie der zusätzlichen Angabe des Jahres, ab welchem der Modulkatalog gelten soll, garantiert. Somit wurde diese Anforderung vollständig erfüllt.\\ \hline %Muss
	\parbox[t]{3cm}	{\textbf{A8}} & Suchen und Filtern im Dozentenpool & Unter dem Navigationspunkt \textit{Dozenten} der Webanwendung findet sich die Funktionalität, in der Liste der Dozenten nach bestimmten Kriterien zu suchen. Diese Anforderung wurde voll umgesetzt.\\ \hline %Muss
	\parbox[t]{3cm}	{\textbf{A9}} & Hinzufügen und Löschen im Dozentenpool & Ebenfalls unter dem Navigationspunkt \textit{Dozenten} ist die Möglichkeit gegeben, Dozenten hinzuzufügen sowie bestehende Dozenten aus der Liste zu entfernen. Diese Anforderung wurde somit umgesetzt. \\ \hline %Muss
	\parbox[t]{3cm}	{\textbf{A10}} & Dozenteninformationen im Dozentenpool & Bei der Auswahl eines Dozenten aus der Liste, wird eine detailliertere Anzeige mit Informationen über diesen eingeblendet. Damit wurde auch diese Anforderung erfüllt.\\ \hline %Muss
	\parbox[t]{3cm}	{\textbf{A11}} & Automatisierte Dozentenanfrage & Eine automatisierte Dozentenanfrage mit vorgefertigtem Template konnte im Rahmen des Projektes nicht umgesetzt werden. Die Möglichkeit, Dozenten über die Anwendung E-Mails zu schicken, ist jedoch insofern gegeben, als dass es die Funktion gibt, dass ein User über einen Dozenten und dessen Mail-Adresse zu seinem jeweiligen Standardmailprogramm weitergeleitet wird und selbbstständig eine E-Mail verfassen kann. Diese als \textit{Kann} priorisierte Anforderung wurde nicht umgesetzt.\\ \hline %Kann
	\parbox[t]{3cm}	{\textbf{A12}} & Zuordnung der Dozenten & Die Umsetzung der Beziehung \enquote{Studiengansleiter-zu-Dozent} des Datenmodells in der Datenbank ordnet die Dozenten jeweils einem Studiengangsleiter zu. Somit konnte diese \textit{Kann}-Anforderung erfüllt werden.\\ \hline %Kann
	\parbox[t]{3cm}	{\textbf{A13}} & Warnungen bei maximaler Stundenanzahl & Diese \textit{Kann}-Anforderung konnte im Rahmen der Umsetzung des Projektes aus zeitlichen Gründen nicht mit umgesetzt werden.\\ \hline %Kann
	\parbox[t]{3cm}	{\textbf{A14}} & Paralleler Zugriff & Die Beziehung \enquote{Kurs-zu-Studiengangsleiter} des Datenmodells ist so umgesetzt worden, dass einem Kurs im notwendigen Anwendungsfall auch zwei Studiengangsleiter zugeordnet werden können. Diese haben dann parallelen Zugriff auf den jeweiligen Kurs. Somit konnte diese \textit{Kann}-Anforderung voll umgesetzt werden.\\ \hline  %Kann
	\parbox[t]{3cm}	{\textbf{A15}} & Tooladministration & Mit der Umsetzung des Navigationspunktes \textit{Administrationsbereich} sowie dem Attribut \texttt{is\_admin} der Entität des Studiengangsleiters wurde diese Anforderung voll umgesetzt.\\ \hline  %Kann
	\parbox[t]{3cm}	{\textbf{A16}} & Kursübersicht & Auf der Hauptansichtsseite der Anwendung wird neben dem Kalender auch eine Übersicht über die Vorlesungen eines Kurses pro Semester inklusive der jeweiligen Prüfungsleistungen dargestellt. Somit konnte diese Anforderung ebenfalls erfüllt werden.\\ \hline  %Kann
	\parbox[t]{3cm}	{\textbf{A17}} & Farbiger Planungsstand & Der Status einer geplanten Vorlesung wird anhand verschiedener Stichworte, welche im Statusfeld eingetragen werden können, in unterschiedlichen Farben dargestellt. Damit wurde auch diese \textit{Kann}-Anforderung umgesetzt.\\ \hline  %Kann
	\parbox[t]{3cm}	{\textbf{A18}} & Planungsexport & Diese \textit{Kann}-Anforderung ist im Rahmen des Projektes aus zeitlichen Gründen nicht umgesetzt worden.\\ \hline  %Kann
	\parbox[t]{3cm}	{\textbf{A19}} & Informationen der Veranstaltungen & Das Anhängen von Dokumenten an einzelne Veranstaltungen eines Kurses wurde im Rahmen des Projektes so nicht mit umgesetzt.\\ \hline  %Kann
	\parbox[t]{3cm}	{\textbf{A20}} & Veranstaltungstitel & Ein Veranstaltungstitel wird wie angefordert in der Form \enquote{Name der Veranstaltung - Name des Dozenten} dargestellt und die Anforderung damit erfüllt.\\ \hline  %Kann
	\parbox[t]{3cm}	{\textbf{A21}} & Automatisierte Benachrichtigungen & Diese \textit{Kann}-Anforderung ist im Rahmen des Projektes aus zeitlichen Gründen nicht umgesetzt worden.\\ \hline  %Kann
	\parbox[t]{3cm}	{\textbf{A22}} & Klausurenmaximum & Diese \textit{Kann}-Anforderung ist im Rahmen des Projektes aus zeitlichen Gründen nicht umgesetzt worden.\\ \hline  %Kann
	\parbox[t]{3cm}	{\textbf{A23}} & Korrekte Moduldurchführung & Diese \textit{Kann}-Anforderung ist im Rahmen des Projektes aus zeitlichen Gründen nicht umgesetzt worden.\\ \hline  %Kann
	\parbox[t]{3cm}	{\textbf{A24}} & Eintragung von Wahlmodulen & Die Umsetzung des Datenmodells mit der Entität der Modulgruppe, welche mehrere Module enthalten kann, garantiert die Möglichkeit, Wahlmodule im Modulkatalog einzutragen. Somit wurde diese Anforderung erfüllt.\\ \hline
	\parbox[t]{3cm}	{\textbf{A25}} & Dozentenvorschläge & Diese \textit{Kann}-Anforderung ist im Rahmen des Projektes aus zeitlichen Gründen nicht umgesetzt worden.\\ \hline  %Kann
	\parbox[t]{3cm}	{\textbf{A26}} & Weiterführbarkeit & Sowohl das Backend als auch das Frontend können künftig erweitert oder modifiziert werden. Diese \textit{Kann}-Anforderung wurde somit erfüllt.\\ \hline %Kann
	\parbox[t]{3cm}	{\textbf{A27}} & Usability & Die Erfüllung dieser subjektiven Anforderung konnte aufgrund der aktuellen Situation mit COVID19 nicht vom Auftraggeber getestet und abgenommen werden. Es wurde jedoch von Projektbeteiligten, welche nicht in der Entwicklung beteiligt waren die Bedienbarkeit der Anwendung getestet und als intuitiv eingeordnet. Daher wird diese Anforderung als erfüllt bewertet.\\ \hline %Muss
	\captionsetup{format=hang}
	\caption{\label{tab:erfanf}Gegenüberstellung der Anforderungen und der Umsetzung \\}
\end{longtable}
