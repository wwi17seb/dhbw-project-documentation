\section{Setup}
\subsection{Repositories von GitHub clonen}

Das Projekt ExoPlan setzt sich aus einem Front-End und einem Back-End zusammen. 
Diese sind jeweils in ein eigenes GitHub-Repository ausgelagert und müssen einzeln ausgecheckt werden. 
Dazu ist es empfehlenswert ein Verzeichnis mit dem Namen \texttt{dhbw-projekt} anzulegen, in welchem die beiden Repositories lokal abgelegt werden. 
Folgende Repositories gilt es auszuchecken:
\begin{itemize}
	\item https://github.com/wwi17seb/dhbw-project-frontend
	\item https://github.com/wwi17seb/dhbw-project-backend
\end{itemize}

Zum Auschecken werden die folgenden Befehle verwendet:

\texttt{git clone https://github.com/wwi17seb/dhbw-project-frontend.git}

\texttt{git clone https://github.com/wwi17seb/dhbw-project-backend.git}

\subsection{Konfiguration vom Back-End}

Um das Back-End verwenden zu können, ist es notwendig \textit{Docker} zu installieren, falls dies noch nicht installiert ist. 
Die korrekte Funktion von Docker kann über die Anzeige der aktuellen Version mit dem folgenden Befehl überprüft werden: \texttt{docker version}.

Wenn \textit{Docker} korrekt funktioniert, sind weitere Konfigurationen notwendig:

\begin{enumerate}
	\item Im Root-Verzeichnis des Back-Ends muss das Verzeichnis \texttt{env} angelegt werden.
	\item Im \texttt{env}-Verzeichnis wird die Datei \texttt{app.properties} mit dem folgenden Inhalt angelegt:
	\begin{lstlisting}
	app.port = 3000
	app.defaultUser = admin
	app.defaultPassword = test
	app.isAdmin = true
	app.forceSync = false
	app.enableTestData = false
	
	server.user     = dhbw
	server.database = becker
	server.password = iH0p3youU5EaSecretPa$$word
	server.port = 5432
	server.dialect = postgres
	jwt.superSecret = TreevgQreNefpuUngHroreunhcg
	AvpugfTrznpug13
	server.host = postgres
	
	pepper = ErarFgvaxg
	\end{lstlisting}
	
	\item Im Verzeichnis \texttt{./docker} wird die Datei \texttt{.env} angelegt. Dort werden weitere Konfigurationen für die Verwendung der Datenbank vorgenommen:
	
	\begin{lstlisting}
	postgres_user=dhbw
	postgres_password=iH0p3youU5EaSecretPa$$word
	postgres_port=5432
	pgadmin_user=project@dhbw.de
	pgadmin_password=test1234
	\end{lstlisting}
	
	\item Im nächsten Schritt müssen die benötigten Zertifikate in das Verzeichnis \texttt{./docker
		/nginx/ssl} kopiert werden.
	
\end{enumerate}

\subsection{Starten von ExoPlan}

Da für das Front-End keine weiteren Konfiguration vorgenommen werden müssen, kann nun das Gesamtsystem gestartet werden. 
Hierfür wird im Back-End-Verzeichnis in das Verzeichnis \texttt{./docker} navigiert und der folgende Befehl ausgeführt:\\ 
\texttt{docker-compose up -{}-build}

Nach einem erfolgreichen Start kann ExoPlan über \textit{https://localhost/} aufgerufen werden. 
Der Administrator kann sich nun mit dem Benutzer \texttt{admin} und dem Passwort \texttt{test} anmelden. Diese Anmeldedaten sollten anschließend sofort geändert werden.

ExoPlan kann des Weiteren über den folgenden Befehl heruntergefahren werden. 
Hierfür muss in das Back-End-Verzeichnis und dem darin befindlichen \texttt{./docker}-Verzeich navigiert werden:

\texttt{docker-compose down}.

Für den Betrieb der Anwendung im Hintergrund sind nähere Informationen in der Dokumentation von Docker gegeben.\footnote{\url{https://docs.docker.com/compose/reference/up/}}