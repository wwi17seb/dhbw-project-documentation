\section{Funktionalitäten}
%TODO Trello-Hinweis über Farben des Status auf Vorlesungspläne: \url{https://trello.com/c/ciMzmZTX}

In diesem Abschnitt soll eine grobe Übersicht über die Hauptfunktionalitäten von Exoplan gegeben werden. Diese sind nachfolgend aufgelistet und werden mit Hilfe von Beispielen erklärt.

Grobe Übersicht über die Hauptprozesse, wie: 
%Keine Screenshots sondern wir läuft der Prozess ab
\begin{itemize}
	\item Neuen Kurs anlegen,
	\item Dozenten kontaktieren (Status nutzen),
	\item Vorlesungen im Kalender verwalten,
	\item Modulkatalog verwalten,
	\item Datenverwaltung - Studiengänge, Schwerpunkte, Prüfungsleistungen,
	\item Administration - Registrierungsschlüssel, Benutzer, Google-Calendar.
\end{itemize}

\subsection{Neuen Kurs anlegen}

Um einen Kurs neuen in Exoplan anzulegen, muss sich der Benutzer mit seinen Anmeldedaten anmelden. Anschließen kann in der Anwendung auf der linken Seite der Menüpunkt \textit{Verlesungspläne} ausgewählt werden. Daraufhin wird der Tab \textit{Kurs hinzufügen} dargestellt, welcher daraufhin geöffnet wird.

Damit das volle Potential des Exoplan-System vollumfänglich ausgeschöpft werden kann, muss die Kurskonfiguration sehr genau durchgeführt werden. Hierbei wird zuerst der Name des Kurses angegeben, also zum Beispiel \textit{WWI17SEB}. Außerdem müssen Studiengang (Wirtschaftinformatik, BWL, usw.) und Studienrichtung (Software Engineering, Sales and Consulting) für den Kurs spezifiziert werden. Abschließend wird die Anzahl der Semester ausgewählt, der Zeitraum für jedes Semester angegeben und die Google Calender ID des Kurses eingetragen.

\subsection{Dozenten kontaktieren}

Im Studienalltag gilt es viele verschiedene Informationen gleichzeitig zu bewältigen. Ein großes Problem ist hierbei eine Überblick über die vorhandenen Dozenten und ihrer Expertise zu behalten. Aus diesem Grund bietet Exoplan eine leistungsstarke und zugleich einfache Dozentenverwaltung. Diese Kann 

\subsection{Vorlesungen im Kalender verwalten}



\subsection{Modulkatalog verwalten}

Auch Modulkataloge lassen sich in Exoplan verwalten, speichern und anpassen. Diese bilden die Grundlage für die Planung von Vorlesungen und gesanten Studienablaufs in Exoplan. Dazu wird der Menüpunkt \textit{Modulkataloge} auf der linken Seite ausgwählt. Anschließend werden alle vorhandenen Modulkataloge angezeigt. Des Weiteren kann ein neuer Modulkatalog durch einen Klick auf den Button \textit{Modulkatalog Hinzufügen} erstellt werden. 

Auch kann ein Modulkatalog durch eine einfache Auswahl geöffnet werden. Hierbei werden anschließen alle dazugehörenden Module angezeigt. Auch lassen sich Module bearbeiten und löschen.

\subsection{Administrationsbereich}

Im Administrationsbereich können verschiedene Einstellungen vorgenommen werden, um den Betrieb von Exoplan zu gewährleisten. Der Administrationsbereich kann durch den Menüpunkt \textit{Administrationsbereich} aufgerufen werden.

\subsubsection{Benutzerverwaltung}

Die Benutzerverwaltung kann über den Tab \textit{Benutzer} erreicht werden und liefert eine Übersicht über alle im System registrierten Benutzer. Es werden Informationen zur Art des Benutzers (Administrator, Studiengangsleiter und Benutzer) und der Passwortstatus (OK, Wechsel erforderlich) angegeben. Des Weiteren kann in dieser Übersicht ein Studiengangsleiter angelegt und auch ein Passwortreset durchgeführt werden.

\subsubsection{Registrierungschlüssel}

Registrierungsschlüssel werden benötigt, damit sich neue Benutzer in Exoplan registrieren können. Dieser Schlüssel kann über den Tab \textit{Registrierungschlüssel} festgelegt werden.

\subsubsection{Google Calender}
