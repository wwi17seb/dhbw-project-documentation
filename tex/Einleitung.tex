\chapter{Einleitung}
Die vorliegende Dokumentation soll einen Überblick über das Projekt, dessen Verlauf sowie die finalen Ergebnisse geben.
In den verschiedenen Kapiteln wird zunächst das Projektziel vorstellt und anschließend der Entwicklungsprozess der erstellten Software erläutert.

Zunächst ist im Kapitel \nameref{ch:Projektmanagement} die Organisation des Projekts und des Projektteams beschrieben.
Anschließend folgt das Einholen der Anforderung und ihre Priorisierung in der \nameref{ch:Anforderungsanalyse}.
Im \nameref{ch:Entwurf} wird sowohl der \nameref{ch:DesignEntwurf} als auch der \hyperref[ch:Technischer Entwurf]{Technische Entwurf} beschrieben.

Die \nameref{ch:Umsetzung} geschildert Infrastruktur des Verwaltungstools, die Umsetzung des Back-Ends sowie Front-Ends und anschließende Tests.
Darüber hinaus wird in einem \nameref{ch:UserGuide} das Aufsetzen der Software und deren Benutzung erläutert.

Schließlich werden in der \nameref{ch:Evaluation} die Anforderungen und deren Umsetzung gegenübergestellt, sowie die Lernerfolge und nächsten Schritte beschrieben.
Abschließend wird ein Fazit zu dem Projektergebnis sowie dem Projektablauf gezogen.
Darüber hinaus wird ein Ausblick bezüglich zukünftigen Entwicklungen gegeben.
Im Anhang sind außerdem vertiefende Inhalte beigefügt.