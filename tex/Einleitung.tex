\chapter{Einleitung}
Zu den Aufgaben eines Studiengangsleiters gehört unter anderem das Planen und Verwalten von Vorlesungen.
Dies ist ein komplexer Vorgang, da geeignete Dozenten gefunden werden müssen und viel Koordination zur Festlegung einzelner Vorlesungstermine notwendig ist.
Da die benötigten Informationen bislang verstreut sind, wird diese Aufgabe dadurch zusätzlich erschwert.
Deshalb soll nun eine Anwendung entwickelt werden, welche alle relevanten Informationen an einer Stelle beinhaltet und das Verwaltung von Vorlesungen vereinfacht.
Durch eine zentrale Anwendung würde dem Studiengangsleiter ein deutlicher Mehrwert entstehen.

Die vorliegende Dokumentation soll einen Überblick über das Projekt, dessen Verlauf sowie die finalen Ergebnisse geben.
In den verschiedenen Kapiteln wird zunächst das Projektziel vorstellt und anschließend der Entwicklungsprozess der erstellten Software erläutert.

Zunächst ist im Kapitel \nameref{ch:Projektmanagement} die Organisation des Projekts und des Projektteams beschrieben.
Anschließend folgt das Einholen der Anforderung und ihre Priorisierung in der \nameref{ch:Anforderungsanalyse}.
Im \nameref{ch:Entwurf} wird sowohl der \nameref{ch:DesignEntwurf} als auch der \hyperref[ch:Technischer Entwurf]{technische Entwurf} beschrieben.

Die \nameref{ch:Umsetzung} schildert die \nameref{ch:Infrastruktur} des Verwaltungstools, die Umsetzung des \hyperref[ch:BackEnd]{Back-Ends} sowie des \hyperref[ch:FrontEnd]{Front-Ends} und abschließende \hyperref[ch:Test]{Tests}.
Darüber hinaus wird in einem \nameref{ch:UserGuide} das Aufsetzen der Software und deren Benutzung erläutert.

Schließlich werden in der \nameref{ch:Evaluation} die Anforderungen und deren Umsetzung gegenübergestellt sowie die Lernerfolge beschrieben.
Abschließend wird ein Fazit zu dem Projektergebnis sowie dem Projektablauf gezogen.
Darüber hinaus wird ein Ausblick bezüglich zukünftiger Entwicklungen gegeben.
Im Anhang sind zusätzlich vertiefende Inhalte beigefügt.