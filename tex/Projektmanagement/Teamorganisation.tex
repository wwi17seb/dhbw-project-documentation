\section{Teamorganisation}
Zur kooperativen Aufgabenbearbeitung erfolgte eine Unterteilung des Kurses in drei dedizierte Teams. Jedes Team unterliegt dabei der Führung eines Teamleiters. Pflicht der Teamleiter ist die Kontrolle und Verteilung von Aufgaben unter den jeweiligen Mitgliedern eines Teams. Des Weiteren sind diese für die Steuerung des Projektes und entsprechende Absprachen verantwortlich. Die Leitung und Durchführung regelmäßiger Meetings innerhalb der Teams sowie des gesamten Kurs erfolgt zugleich durch die Projektleitung. Die Unterteilung in Teams gestaltet sich wie folgt:
\begin{enumerate}
    \item Produktmanagement (Leitung: Kay Wessel)
    \item Frontend (Leitung: Sandra Keller)
    \item Backend (Leitung: Martin Sandig)
\end{enumerate}

Die Organisation zur Realisierung des Projektes erfolgte mittels der Nutzung unterschiedlicher digitaler Medien. 
Dabei wird dem  Aufgaben-Verwaltungs-Onlinedienst \textit{Trello}\footnote{Trello.com} eine essentielle Rolle zuteil. In der Anwendung ist es möglich, in sogenannten Boards Aufgaben zu verwalten. Die Aufgaben können beliebig bearbeitet werden und mit Checklisten, Anhängen, Terminen und vielem mehr versehen werden.
Für jedes Team wurde in Trello ein eigenständiges Board erstellt, sodass die Aufgaben im Verbund eines Teams entsprechend zugeordnet und durch jeweilige Mitglieder bearbeitet werden können. Zur Klärung allgemeiner Fragen wurde ein gemeinsames Board \glqq Organisation\grqq eingerichtet. Da für das Projekt eine Protokollierung der Stunden erforderlich ist wurde ein Excel-Liste erstellt. Diese wird durch die Projektleitung gepflegt und zeigt somit den aktuellen Stundenkontingent der jeweiligen Teammitglieder auf.
Die Kommunikation und Koordination erfolgte über den WhatsApp-Messenger, sowie das Konferenzsystem Discord. Hierdurch konnten auditiv Meetings abgehalten sowie kommunikativ Aufgaben im Verbund bearbeitet werden. 