\section{Test}
\label{ch:Test}
Aufgrund der begrenzten Zeit wird während der Entwicklung hauptsächlich manuell getestet. 
Als weitere Maßnahme zur Qualitätssicherung werden Code Reviews eingesetzt.
Neben Pair-Programming sowie Walkthroughs werden hauptsächlich tool-basierte Code Reviews genutzt, indem bei GitHub per Pull Request Feedback von geeigneten anderen Entwicklern zu jeder Codeänderung eingefordert wird.
Dadurch kann die Codequalität sichergestellt sowie Feedback zu den Funktionalitäten eingeholt werden.

Nach Beendigung der Entwicklung wurde außerdem im Front-End ein Abnahmetest von dem Projektergebnis durch zwei Projektmitglieder durchgeführt, um die Funktionalitäten zu prüfen sowie einen Abgleich der Benutzeroberfläche durchzuführen. 
Hierbei wurden die Referenzen Design Entwurf aus Kapitel \vref{ch:DesignEntwurf}, der Styleguide im Anhang \vref{an:Styleguide} sowie die Anforderungen der Tabelle \vref{tab:Anforderungen} hinzugezogen.
Dabei wurden mehrere geringfüge Fehler gefunden und direkt verbessert.
Zusätzlich wurden Aspekte festgehalten, die zeitlich nicht mehr geändert sowie umgesetzt werden, jedoch für eine Weiterentwicklung des Produkts von Relevanz sind: 

\begin{itemize}
    \item Erlaubte Formate und maximale Dateigröße von hochgeladenen Dokumenten angeben und überprüfen
    \item Löschen von Modulkatalogen 
    \item Entziehen von Administrationsrechten
    \item Studiengangsrichtung nur bei zutreffenden Studiengängen anzeigen (zum Beispiel der Studiengang Digitale Medien hat keine Studienrichtungen)
    \item Hinzufügen von Vorlesungsinformationen zu den Kalendareinträgen.
\end{itemize}
