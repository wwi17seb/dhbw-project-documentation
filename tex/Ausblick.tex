\chapter{Fazit und Ausblick}
\section{Fazit}
Abschließend lässt sich sagen, dass das Projekt sehr erfolgreich abgeschlossen werden konnte. 
Wie in Tabelle \vref{tab:erfanf} ersichtlich wurden alle Muss-Anforder"-ungen und circa die Hälfte der Kann-Anforderungen erfüllt.
Bei dem Projektteam war allgemein eine gute Arbeitsmoral von Projektbeginn bis -ende spürbar.
Jedes Teammitglied konnte und hat seine Fähigkeiten einbringen können und etwas zu dem Projekterfolg beigetragen.
Dies ist auch aus den Bewertungspunkten ersichtlich,die kaum Ausreißer nach unten aufweisen.
Während des Projekts konnten bereits mehrere Dinge verbessert werden. 
Dazu zählt die Zeitplanung des 6. Semesters, welche zu den jeweiligen zeitlichen Abschnitten inhaltliche Ziele beinhaltet, sodass der Fortschritt der Entwicklung besser überprüft werden konnte.
Eine weitere Verbesserung bestand in der Kommunikation.
Es wurde generell verstärkt kommuniziert, sowohl im gesamten Team, den jeweiligen Untergruppen als auch mit den Leitern des Projekts. 
Wichtig zu erwähnen ist auch, dass das durch COVID-19 bedingte Online-Semester kein Hindernis für die Umsetzung des Projekts darstellte.
Die Absprachen sowie das kollaborative Entwickeln konnten über das verwendete Konferenzsystem verteilt stattfinden.

\section{Ausblick}
Das Endprodukt bietet noch Möglichkeiten zur weiteren Entwicklung. 
Im Folgenden werden mögliche Schritte zur Erweiterung aufgeführt.
Dazu zählen hauptsächlich die noch nicht umgesetzten Kann-Anforderungen, welche der Tabelle \vref{tab:erfanf} entnommen werden können.
Außerdem wurden durch den Abnahmetest in Kapitel \vref{ch:Test} zusätzliche Möglichkeiten zur Erweiterung aufgezeigt.

Die im Folgenden beschriebenen Punkte sind ebenfalls für die zukünftige Verwendung des Tools zu berücksichtigen.
Für die Entwicklung der Anwendung wurden selbstsignierte Zertifikate verwendet, die vor der Verwendung ausgetauscht werden sollten.
Außerdem sind nicht alle entwickelten Routen komplett nach dem \ac{REST}-Paradigma erstellt, wodurch dort eine weitere Möglichkeit zur Verbesserung besteht.
Für den laufenden Betrieb ist außerdem zu beachten, dass die Datenbank in einem Docker-Container läuft und deshalb nur so lange persistiert ist, wie der Container ohne Unterbrechung hochgefahren ist.
Aus diesem Grund sollten regelmäßige Backups durchgeführt oder der Container auf eine persistente Schicht gemountet werden. 
Des Weiteren werden Passwörter in der Anwendung als Hashwert gespeichert.
Eine Verbesserungsmöglichkeit ist den Hashwert direkt Clientseitig zu bilden, damit Passwörter nie im Klartext an den Server übermittelt werden.